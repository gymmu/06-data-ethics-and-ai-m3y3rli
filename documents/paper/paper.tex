\documentclass{report}

\usepackage[ngerman]{babel}
\usepackage[utf8]{inputenc}
\usepackage[T1]{fontenc}
\usepackage{hyperref}
\usepackage{csquotes}
\usepackage[a4paper]{geometry}

\usepackage[
    backend=biber,
    style=apa,
    sortlocale=de_DE,
    natbib=true,
    url=false,
    doi=false,
    sortcites=true,
    sorting=nyt,
    isbn=false,
    hyperref=true,
    backref=false,
    giveninits=false,
    eprint=false]{biblatex}
\addbibresource{../references/bibliography.bib}


\title{Ethik im Umgang mit Daten}
\author{Linus Meyer}
\date{\today}


\begin{document}

\maketitle



\tableofcontents

\chapter{Einleitung}

Die KI ist eine Technologie, die unseren digitalen Wandel und die Gesellschaft vorantreibt. 
Die KI versucht das menschliche Denken zu imitieren und wird immer besser darin. Von Menschen und von einer KI geschriebene Texte können fast kaum mehr unterschieden werden. 
Durch Maschinelles Lernen, natürliche  Sprachverarbeitung und mehr kann die KI sich weiterentwickeln und Aufgaben automatisiert erledigen.
Im folgenden Aufsatz werde ich die ethischen Prinzipien im Zusammenhang mit Daten und der KI behandeln.

\newpage

\chapter{Definitionen}

KI:
Daten:

\newpage
\section{Etwas mit Quellen}

Etwas mit Änderung hier am Ende.

Wenn ich eine Quelle zitieren möchte, kann ich das ganze einfach am Ende des Satzes machen \citep{example}. Oder wie \citet{example} sagt, auch mitten im Text.

\printbibliography

\end{document}
