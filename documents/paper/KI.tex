\chapter{Der Einsatz künstlicher Intelligenz im Alltag} 
Der Einsatzbereich künstlicher Intelligenz im Alltag ist riesig und wird weit mehr verwendet als gedacht. In den letzten Jahren hat die Verwendung erheblich zugenommen. Hier sind die positiven, sowie negative Aspekte der KI:
\section{Postive Aspekte}
\subsection{Medizinische Anwendungen}
Künstliche Intelligenz kann die Gesundheitsversorgung deutlich verbessern. Durch die Analyse großer Datenmengen können Systeme der künstlichen Intelligenz Ärzten helfen, Diagnosen zu stellen und Behandlungen zu planen. Beispielsweise können KI-gestützte Bildgebungstechniken dabei helfen, Anomalien in medizinischen Bildern wie Röntgen- und MRT-Scans zu erkennen, was zu früheren und genaueren Diagnosen führen kann. Dazu ermöglichen Algorithmen der künstlichen Intelligenz die Erstellung individueller Behandlungspläne entsprechend den individuellen Bedürfnissen des Patienten.
\subsection{Finanzanwendungen}
In der Wirtschaft fördert künstliche Intelligenz die Automatisierung und Optimierung von Geschäftsprozessen. Unternehmen nutzen KI, um Markttrends vorherzusagen, Lieferketten zu rationalisieren und maßgeschneiderte Marketingstrategien zu entwickeln. Beispielsweise nutzen Banken künstliche Intelligenz, um Kreditrisiken einzuschätzen und Betrug aufzudecken, was die Sicherheit und Effizienz von Firmen erhöht. Online-Shops nutzt ausserdem KI-basierte Empfehlungssysteme, um Kunden personalisierte Produktempfehlungen zu geben, die den Umsatz steigern und die Kundenzufriedenheit verbessern können.
\subsection{Alltagsanwendungen}
KI-basierte Technologien erleichtern viele Aufgaben im Alltag. Sprachassistenten wie Siri, Google Assistant und Alexa ermöglichen es Benutzern, Informationen zu empfangen, Aufgaben zu organisieren und Smart-Geräte zu Hause mit Sprachbefehlen zu steuern. KI-gestützte Systeme verbessern zudem das Online-Einkaufserlebnis, indem sie personalisierte Produktempfehlungen und Suchergebnisse bereitstellen. Darüber hinaus helfen auf künstlicher Intelligenz basierende Navigationssysteme dabei, die effizientesten Routen zu finden.
\section{Negative Aspekte}
\subsection{Diskriminierung und Vorurteile}
Ein großes Problem beim Einsatz künstlicher Intelligenz ist die Möglichkeit, Vorurteile und Diskriminierung zu verstärken. KI-Algorithmen können unwissentlich rassistische, geschlechtsspezifische oder andere diskriminierende Vorurteile in Trainingsdaten übernehmen. Untersuchungen haben gezeigt, dass KI-Systeme häufig Geschlechterstereotypen, Rassismus und Homophobie reproduzieren können. Beispielsweise können Gesichtserkennungssysteme bei der Identifizierung von Menschen unterschiedlicher ethnischer Herkunft ungenau sein, was zu ungerechtfertigten Polizeikontrollen oder Diskriminierung führen kann.
\subsection{Arbeitsplatzverluste}
KI-Automatisierung kann auch zu Arbeitsplatzverlusten führen, insbesondere in Branchen, in denen wiederholende Aufgaben häufig sind. Künstliche Intelligenz wird in vielen Bereichen die Produktivität und Effizienz steigern, aber in weniger qualifizierten Berufen könnten Arbeitskräfte durch Maschinen ersetzt werden, was zu Arbeitslosigkeit führen könnte. Um dies zu verhindern oder den betroffenen Personen entgegenzuwirken, sollten Umschulungen und neue Arbeitsplätze angeboten werden.
\subsection{Datenschutz und Privatsphäre}
Ein weiterer kritischer Faktor beim Einsatz künstlicher Intelligenz ist die Verarbeitung personenbezogener Daten. Künstliche Intelligenzsysteme sammeln und analysieren große Datenmengen, um effizient zu arbeiten, was kritisch in Bezug des Datenschutzes und der Privatsphäre ist. Es besteht die Gefahr, dass sensible Informationen missbraucht oder unzureichend geschützt werden, was zu Datenschutzverletzungen und Identitätsdiebstahl führt.
