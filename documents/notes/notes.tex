\documentclass{article}

\usepackage[ngerman]{babel}
\usepackage[utf8]{inputenc}
\usepackage[T1]{fontenc}
\usepackage{hyperref}
\usepackage{csquotes}

\usepackage[
    backend=biber,
    style=apa,
    sortlocale=de_DE,
    natbib=true,
    url=false,
    doi=false,
    sortcites=true,
    sorting=nyt,
    isbn=false,
    hyperref=true,
    backref=false,
    giveninits=false,
    eprint=false]{biblatex}
\addbibresource{../references/bibliography.bib}

\title{Notizen zum Projekt Data Ethics}
\author{Linus Meyer}
\date{\today}

\begin{document}
\maketitle

\section{Notizen}
\section{Einleitung} 

- KI: bedeutende Technologie, beeinflusst tägliches Leben
- Anwendungen: von Medizin bis Wirtschaft
- Untersuchung: Definition, Nutzung im Alltag, ethische Fragen

\section{ Definition}

- Fähigkeit von Computern/Maschinen, menschliche Aufgaben auszuführen
- Problemlösung, Lernen, Mustererkennung, Entscheidungsfindung
- Nutzung großer Datenmengen, komplexe Modelle

\section{Einsatz im Alltag} 

\subsection{ Positive Aspekte:}
  - Medizin: Verbesserte Gesundheitsversorgung, Diagnose, Behandlung
  - Finanzen: Automatisierung, Markttrends, Betrugserkennung
  - Alltag: Sprachassistenten, personalisierte Empfehlungen, Navigation
  
\subsection{Negative Aspekte:} 
  - Diskriminierung: Vorurteile in Daten, ungenaue Gesichtserkennung
  - Arbeitsplatzverluste: Automatisierung, Arbeitslosigkeit in bestimmten Berufen
  - Datenschutz: Verarbeitung personenbezogener Daten, Datenschutzverletzungen

\subsection{Entwicklung einer KI-Ethik} 

- Förderung verantwortungsvoller KI-Nutzung
- Organisationen: UNESCO, Europäisches Parlament, ethische Richtlinien
- Unternehmen: IBM, ethische KI-Praktiken

\subsection{Fazit}

- KI: Potenzial zur Verbesserung, aber auch ethische Herausforderungen
- Verantwortungsvolle Nutzung notwendig für positive gesellschaftliche Veränderungen




\end{document}
